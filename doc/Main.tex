\documentclass[letterpaper, 12pt, twosided, twocolumn]{article}
\usepackage{fullpage}

\title{A Networked Software Audio Mixing Console}
\author{Jeffrey M. Clark}
\date{\today}

\begin{document}

\maketitle

\section{Introduction}

This project describes a networked software audio mixing console (N-SAMC). Most mixing consoles for live audio are hardware-based, which ensures a high degree of stability within the system -- often desirable for real-time audio processing. Software mixers are often used in conjunction with a digital audio workstation (DAW), and are characterized by their flexibility.

This project, \textit{prima facie}, appears to take the worst of both world: the relative inflexibility of a hardware console with the instability of the software console. However, the this console is not specifically intended for live-audio work; but rather for testing and development of related systems which need a specifically-known basis. Additionally, the N-SAMC incorporates an OSC structure that is open, known, and documented. For development and testing, it is helpful to have a system that is completely known.

\section{Signal Flow}

The N-SAMC emulates the traditional mixing structure of a console mixer. The base implimentation consists of a dynamics range processor, a multiband, parametric equalizer, a low-cut filter, a stereo panner, and a fader. The mixer also includes 6 auxillary sends, which are switchable pre or post fader.

\subsection{Low-Cut Filter}

The low-cut filter occurs first in the signal chain. It is built from a pair of cascaded biquad filters, emulating a fourth-order IIR filter topology.

\subsection{Dynamics Range Processing}

The dynamics range processing section consists of two processing units. Either unit can function as either a compressor or a gate. The position of the dynamics range processor is switchable to pre or post-EQ.

The settings follow the usual \emph{attack}, \emph{release}, and \emph{threshold} values. The module also includes visual meters that monitor the input gain and the gain reduction values.

\subsection{Equalizer}

The equalizer section consists of a bank of four filters which are settable parametrically by the user throught the usual \emph{frequency}, \emph{Q}, and \emph{gain} values. The current implimentation fixes the filters as peaking filters.

\subsection{Fader}

The fader section includes a fader. The fader occurs pre-panner in signal-flow, despite visually ocurring below the panning controls. The fader applies a simple multiplier against the input signal.

\subsection{Panner}

The panning module is a simple stereo panner. 

\end{document}